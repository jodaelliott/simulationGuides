\documentclass{beamer}

\usepackage{graphicx}
\usetheme{Madrid}

\title{
\textsc{Scarf}
}

\subtitle[]{
Guide to running simulations for DLS Spectroscopy
}

\author[J D Elliott]{
\footnotesize
Joshua D Elliott
}

\institute[DLS Ltd.]{
Diamond Light Source Ltd.
}

\date[V0.1 2022]{
\footnotesize
Updated: \today
}

\begin{document}

   \frame{\titlepage}

   \frame{
      \frametitle{Access to \textsc{Scarf}}
      \footnotesize

      \begin{alertblock}{
      \textbf{Login}:  ssh -Y fedid@ui1.scarf.rl.ac.uk} 

      \begin{itemize}

         \item ssh: secure shell, log in to a remote machine (scarf)
         \item -Y: X11 forwarding, access graphical interfaces
         \item fedid: Federal ID and username to access scarf
         \item ui1.scarf.rl.ac.uk: host address

      \end{itemize}

      \end{alertblock}
   }

   \frame{
      \frametitle{Places on \textsc{Scarf}}
      \footnotesize

       \begin{center}
          \begin{tabular}{l l l}
             \textbf{Name} & \textbf{Location} & \textbf{Description} \\ \hline
             Home & \$HOME & 80 GB, weekly back up\\
             Scratch & /scratch/ & Area for simulations (not recommended, randomly purged)\\
                          & /scratch21/ & Area for simulations (not recommended, randomly purged)\\
             Work & /work4/dls/ & Area for simulations (recommended not backed up)\\ \hline
       \end{tabular}
       \end{center}
   
   }

      \frame{
      \frametitle{Useful Commands 1}
      \framesubtitle{Bash}
      \tiny

       \begin{center}
          \begin{tabular}{l l l l}
             \textbf{Command} & \textbf{Description} \\ \hline
             & & & \\
             \textbf{Files} & & \textbf{Modules}\\
             ls -ltrh                 & list directory, list format                            & module list & list loaded modules\\ 
             cat                      & display contents of file                             & module avail & list available modules \\
             cd                       & change to home directory                        & module load & load module \\
             cd ../                   & change to upper directory                        & module purge & clear all loaded modules \\
             cd /\emph{path}/ & change to specific directory path             & module show & display module information \\
             cp.                      & copy file a to b                                         & & \\
             cp -r                    & copy recursively a to b (folders)              & \textbf{Git} & \\
             mkdir                  & make a directory                                      & git add                                      & add files/folders to committed \\
             mv                      & move folder a to b                                    & git branch                                 & display all branches \\
             rm                       & remove a file                                            & git checkout                             & checkout a branch \\ 
             rm -r                   & remove recursively (deleting folders)       & git checkout -b                         & create a branch and checkout \\
                                       &                                                                   & git clone                                   & clone a repository \\
             \textbf{Batch Jobs} &                                                              & git commit -m                           & package changes\\
             sbatch job.sh & submit job.sh to scheduler                            & git pull                                      & update local branch\\
             squeue & display full queue                                                    & git push -{}-set-upstream origin & update remote branch\\
             squeue -u scarfxxxx & display user xxxx queued jobs            & git status                                 & show changes  \\
             scancel xxxxx & cancel job xxxxx & & \\
       \end{tabular}
       \end{center}
   
   }
   
   \frame{
      \frametitle{Useful Commands 2}
      \framesubtitle{vim}
      
   }
   
   \frame{
   \frametitle{Getting Started with Quantum Espresso 1}
   \framesubtitle{Steps to running Quantum Espresso on \textsc{Scarf} for the first time}
   \tiny

   \begin{alertblock}{Make directory for simulations}
      Make a directory under work4 and create a symbolic link in home directory to access.
      \begin{itemize}
        \item cd /work4/dls
        \item mkdir \$fedid
        \item cd
        \item ln -s /work4/dls/\$fedid .   
      \end{itemize}      
   \end{alertblock}
   
   \begin{alertblock}{Obtain example files}
      Create a folder for example simulations and download the example files directly from the github repository.
      \begin{itemize}
         \item cd \$fedid
         \item mkdir test
         \item cd test
         \item git clone https://github.com/jodaelliott/XSpectra-CONEXSworkshop2022.git
         \item cd XSpectra-CONEXSworkshop2022
      \end{itemize}
   \end{alertblock}
   
   }
   
   \frame{
   \frametitle{Getting Started with Quantum Espresso 2}
   \framesubtitle{Steps to running Quantum Espresso on \textsc{Scarf} for the first time}
   \tiny

   \begin{alertblock}{Load modules for simulations and plotting}
      Search and load Quantum Espresso (pw.x, xspectra.x) and xmgrace modules for simulations and plotting the results.
      \begin{itemize}
        \item module avail espresso
        \item module load contrib/dls-spectroscopy/quantum-espresso/6.5-intel-18.0.3
        \item module avail grace
        \item module load xmgrace/5.1.25
        \item module list   
      \end{itemize}      
   \end{alertblock}
   
   \begin{alertblock}{Run first calculations}
      Execute diamond.scf.in and diamond.xspectra.in by editing job.sh file and using upf2core.sh script
      \begin{itemize}
         \item cd 01\_Diamond
         \item sbatch job.sh
         \item squeue -u scarfxxxx
         \item vi job.sh; edit input=diamond.xspectra and mpirun command to xspectra.x
         \item ../tools/upf2plotcore.sh ../pseudopotentials/C\_PBE\_TM\_2pj.UPF $>$ C.wfc
         \item sbatch job.sh
         \item xmgrace xanes.dat
      \end{itemize}
   \end{alertblock}
   
   }
   
   %%%%
   % diff
   % The steps required to run a calculation

\end{document}

