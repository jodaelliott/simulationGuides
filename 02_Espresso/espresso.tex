\documentclass{beamer}

\usepackage{amsmath}
\usepackage{graphicx}
\usetheme{Madrid}

%\setlength{\parskip}{\baselineskip}

\AtBeginSection[]{
  \begin{frame}
  \vfill
  \centering
  \begin{beamercolorbox}[sep=8pt,center,shadow=true,rounded=true]{title}
    \usebeamerfont{title}\insertsectionhead\par%
  \end{beamercolorbox}
  \vfill
  \end{frame}
}

\title{
Quantum \textsc{Espresso}
}

\subtitle[]{
Guide to running simulations for DLS Spectroscopy
}

\author[J D Elliott]{
\footnotesize
Joshua D Elliott
}

\institute[DLS Ltd.]{
Diamond Light Source Ltd.
}

\date[V0.1 2022]{
\footnotesize
Updated: \today
}

%\begin{frame}
%\footnotesize
%\frametitle{}
%\framesubtitle{}
%\end{frame}

%%% OUTLINE %%%

% 1. Introduction to Quantum Espresso Code
%    - Concept
%    - Licence
%    - Help
%    - What Quantum Espresso is
% 2. Density Functional Theory
%    - QM with Density 
%    - KS Equation
%    - Self consistency cyle
%    - Self consistency in the output
%    - Representing quantities on a computer

\begin{document}

\frame{\titlepage}

\section{Introduction}

\subsection{What is Quantum \textsc{Espresso}}

\begin{frame}
\footnotesize
\frametitle{Quantum \textsc{Espresso}}
\framesubtitle{opEn Source Package for Research in Electronic Structure
Simulations and Optimization}
\end{frame}

\begin{frame}
\footnotesize
\frametitle{Open Source Software Package}



\end{frame}


\begin{frame}
\footnotesize
\frametitle{Where to get help?}

\alert{\textbf{At Diamond}}:\\
Email: joshua.elliott@diamond.ac.uk; Office: 1.12 Z02 (Ring); Ext:\\
Email: mihai.duta@diamond.ac.uk (installation/compilation problems)\\~\\

\textbf{Online}:\\
Web-site: www.quantum-espresso.org\\~\\

\textbf{Literature}:\\
Journal of Physics: Condensed Matter 2009, 21 (39), 395502.\\
Journal of Physics: Condensed Matter 2017, 29 (46), 465901.\\
Journal of Chemical Physics 2020, 152 (15), 154105.\\~\\

\textbf{Source Code}:\\
Doc folder under quantum espresso distribution\\~\\

\textbf{Mailing List}:\\
Email: users@lists.quantum-espresso.org\\
Archive of questions/issues dating back to 2011.\\
Web-site: https://www.mail-archive.com/users@lists.quantum-espresso.org/
\end{frame}

\begin{frame}
\footnotesize

\frametitle{A Software Suite}

Important: QE is not a single executable file, rather a collected distribution of several
executable programs.\\

\begin{center}
\begin{minipage}{0.85\textwidth}
\begin{tiny}
\begin{itemize}
   \item[\alert{PWscf}] Ground state electronic structure, structural optmiziation.
   \item[CP] Carr-Parrinello molecular dynamics.
   \item[PHonon] Linear-repsonse calculations.
   \item[\alert{PostProc}] Post processing, graphs and visualization.
   \item[PWneb] Nudged-Elastic Band driver from reaction paths.
   \item[atomic] Generation of pseudopotentials.
   \item[PWGui] Graphical User Interface for input.
   \item[PWcond] Ballistic conductance calculations.
   \item[\alert{XSpectra}] Core-level excitation spectra based on Fermi-Golden Rule.
   \item[GWL] Quasiparticle and Exciton energies based on GW/BSE approximation.
   \item[TD-DFPT] Time-dependent Density Functional Perturbation Theory.
   \item[EPW] Electron-phonon coupling.
   \item[HP] Automation of linear-response DFT+U parameters. 
\end{itemize}
\end{tiny}
\end{minipage}
\end{center}

All these packages and others share: (i) installation method, (ii) input file format,
(iii) pseudopotential file format, (iv) output format (v) source code.

\end{frame}

\section{Density Functional Theory - Theory}

\begin{frame}
\footnotesize
\frametitle{Quantum mechanics in terms of the electron density}

\onslide<1->{The total energy of the ground state of a system of electrons may be written as a
functional of the electron density.}

\begin{equation}
\onslide<1->{E^\mathrm{DFT}[n]; \hspace{2cm} 
\int n(\mathbf{r})\ \mathrm{d}\mathbf{r} = N; \hspace{2cm} 
n(\mathbf{r}) \ge 0}
\end{equation}

\onslide<2->{We can prove mathmatically that the functional exists, but the form is unknown. Within
Kohn-Sham formalism we write the functional:}

\onslide<2->{
\begin{equation}
E^\mathrm{DFT}[n] = T_s[\{\psi_i\}] + E_\mathrm{ext}[n] + E_\mathrm{Hartree}[n] +
E_{xc}[n] + E_\mathrm{ions}
\end{equation}
}

\onslide<3->{Within the functional we introduced the set of single particle orbitals $\{\psi_i\}$,
which are related to the density} 

\onslide<3->{
\begin{equation}
n(\mathbf{r}) = \sum_i^M \vert \psi_i(\mathbf{r}) \vert^2
\end{equation}
}

\onslide<4->{These are used to approximate the kinetic energy}
\onslide<4->{
\begin{equation}
T_s[\{\psi_i\}] = -\frac{1}{2} \sum_i^N \int \mathrm{d}\mathbf{r}\ \psi_i^*(\mathbf{r})
\nabla^2 \psi_i(\mathbf{r})
\end{equation}
}

\end{frame}

\begin{frame}
\footnotesize
\frametitle{Quantum mechanics in terms of the electron density}

\onslide<1->{The other terms are functionals of the density:

\begin{equation}
E_\mathrm{ext}[n] = \int \mathrm{d}\mathbf{r}\ n(\mathbf{r})V_\mathrm{ext}(\mathbf{r}),
\end{equation}

\begin{equation}
E_\mathrm{Hartree}[n] = \frac{1}{2} \int \mathrm{d}\mathbf{r} \mathrm{d}\mathbf{r}^\prime\
\frac{n(\mathbf{r})n(\mathbf{r}^\prime)}{\vert \mathbf{r} - \mathbf{r}^\prime \vert},
\end{equation}
}

\onslide<2->{The Ionic term is the pairwise interaction of the charged nuclei

\begin{equation}
E_\mathrm{ions} = \sum_{I,J\ne I} \frac{Z_I Z_J}{\vert \mathbf{R}_I - \mathbf{R}_J \vert}
\end{equation}

There difficulty with the approach lies in the exchange-correlation energy, whose form
remains unknown.
}

\onslide<3->{
Minimization of the functional derivitive of $E^\mathrm{DFT}$ with respect to the density
$n$ gives

\begin{equation}
H^\mathrm{KS} \psi_i(\mathbf{r}) = 
\left [ 
-\frac{1}{2}\nabla^2 + V_\mathrm{ext}(\mathbf{r}) + V_\mathrm{Hartree}(\mathbf{r}) +
V_{xc}(\mathbf{r}) 
\right ] \psi_i(\mathbf{r}) = \varepsilon_i \psi_i(\mathbf{r})
\end{equation}
}
\end{frame}

\begin{frame}
\footnotesize
\frametitle{The Kohn-Sham Equations}

\onslide<1->{
\begin{equation}
H^\mathrm{KS} \psi_i(\mathbf{r}) = 
\left [ 
-\frac{1}{2}\nabla^2 + V_\mathrm{ext}(\mathbf{r}) + V_\mathrm{Hartree}(\mathbf{r}) +
V_{xc}(\mathbf{r}) 
\right ] \psi_i(\mathbf{r}) = \varepsilon_i \psi_i(\mathbf{r})
\end{equation}
}

\onslide<2->{
\begin{tiny}
\begin{equation}
V_\mathrm{Hartree} = 
\int \mathrm{d}\mathbf{r}^\prime\ 
\frac{n(\mathbf{r}^\prime)}{\vert \mathbf{r} - \mathbf{r}^\prime \vert } 
\end{equation}
\begin{equation}
V_{xc} = \frac{\delta E_{xc}}{\delta n}
\end{equation}
\end{tiny}
}

\end{frame}
\end{document}
